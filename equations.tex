\documentclass{article}
\usepackage{amsmath}
\usepackage{xcolor}

\begin{document}

\section{Equations}
 (from “Triangular” ideas. Idea 03, “The disk”. Version 010.3)

Red terms indicate a 2 dimensional system. Removing them would bring the system to 1D.

$$\ddot{x}_0 =-\frac{g}{m}\frac{-n_{1/2}}{x_{0}-x_{1/2}}$$

$$\ddot{x}_i =-\frac{g}{m}\frac{n_{i-1/2}-n_{i+1/2}}{x_{i-1/2}-x_{i+1/2}}$$

$$0<x_N<x_{N-1}<...<x_1<x_0<\infty $$

with

\begin{equation}
  \begin{split}
    x_{i-1/2} & \equiv \frac{x_{i-1}+x_i}{2} \quad\text{for}\quad 0<i \\
    x_{i+1/2} & \equiv \frac{x_i+x_{i+1}}{2} \quad\text{for}\quad i<N\\
    x_{N+1/2} & \equiv \frac{x_N}{2}\\
    n_{iª-1/2} & \equiv n^{\delta t}\left(x^{\delta t}_{i-1/2}\right)\textcolor{red}{\frac{x^{\delta t}_{i-1/2}}{x_{i-1/2}}}\frac{x^{\delta t}_{i-1}-x^{\delta t}_i}{x_{i-1}-x_i}\\
    n_{i+1/2} & \equiv n^{\delta t}\left(x^{\delta t}_{i+1/2}\right)\textcolor{red}{\frac{x^{\delta t}_{i+1/2}}{x_{i+1/2}}}\frac{x^{\delta t}_i-x^{\delta t}_{i+1}}{x_i-x_{i+1}}\quad\text{for}\quad i<N\\
    n_{N+1/2} & \equiv n^{\delta t}\left(x^{\delta t}_{N+1/2}\right)\textcolor{red}{\frac{x^{\delta t}_{N+1/2}}{x_{N+1/2}}}\frac{x^{\delta t}_N}{x_N}
  \end{split}
\end{equation}



Forces can be rewritten as:

\begin{equation}
  \begin{split}
    \ddot{x}_i & =-\frac{g}{m}\frac{n_{i-1/2}-n_{i+1/2}}{x_{i-1/2}-x_{i+1/2}}\\
    & = -\frac{g}{m}\frac{n^{\delta t}\left(x^{\delta t}_{i-1/2}\right)\textcolor{red}{\frac{x^{\delta t}_{i-1/2}}{x_{i-1/2}}}\frac{x^{\delta t}_{i-1}-x^{\delta t}_i}{x_{i-1}-x_i}-n^{\delta t}\left(x^{\delta t}_{i+1/2}\right)\textcolor{red}{\frac{x^{\delta t}_{i+1/2}}{x_{i+1/2}}}\frac{x^{\delta t}_i-x^{\delta t}_{i+1}}{x_i-x_{i+1}}}{x_{i-1/2}-x_{i+1/2}}\\
    & = -2\frac{g}{m}\frac{n^{\delta t}\left(\frac{x^{\delta t}_{i-1}+x^{\delta t}_i}{2}\right)\textcolor{red}{\frac{x^{\delta t}_{i-1}+x^{\delta t}_i}{x_{i-1}+x_i}}\frac{x^{\delta t}_{i-1}-x^{\delta t}_i}{x_{i-1}-x_i}-n^{\delta t}\left(\frac{x^{\delta t}_{i+1}+x^{\delta t}_i}{2}\right)\textcolor{red}{\frac{x^{\delta t}_{i}+x^{\delta t}_{i+1}}{x_{i}+x_{i+1}}}\frac{x^{\delta t}_i-x^{\delta t}_{i+1}}{x_i-x_{i+1}}}{x_{i-1}-x_{i+1}}\\
    & = \frac{\Phi_i^-}{\textcolor{red}{\left(x_{i-1}+x_i\right)}\left(x_{i-1}-x_i\right)\left(x_{i-1}-x_{i+1}\right)}+\frac{\Phi_i^+}{\textcolor{red}{\left(x_i+x_{i+1}\right)}\left(x_i-x_{i+1}\right)\left(x_{i-1}-x_{i+1}\right)}
  \end{split}
\end{equation}


where

\begin{equation}
  \begin{split}
    \Phi_i^-&= -2\frac{g}{m}n^{\delta t}\left(\frac{x^{\delta t}_{i-1}+x^{\delta t}_i}{2}\right)
    \textcolor{red}{\left(x^{\delta t}_{i-1}+x^{\delta t}_i\right)}\left(x^{\delta t}_{i-1}-x^{\delta t}_i\right) \\
    \Phi_i^+&=+2\frac{g}{m}n^{\delta t}\left(\frac{x^{\delta t}_{i+1}+x^{\delta t}_i}{2}\right)\textcolor{red}{\left(x^{\delta t}_{i}+x^{\delta t}_{i+1}\right)}\left(x^{\delta t}_{i}-x^{\delta t}_{i+1}\right)
  \end{split}
\end{equation}


are fixed parameters not depending on $x$ and, thus, can be computed beforehand.

\section{Boundary conditions}
\subsection{$i=0$}

\begin{equation}
  \begin{split}
    \ddot{x}_0 &=-\frac{g}{m}\frac{-n_{1/2}}{x_{0}-x_{1/2}}\\
    &=-\frac{g}{m}\frac{-n^{\delta t}\left(x^{\delta t}_{1/2}\right)\textcolor{red}{\frac{x^{\delta t}_{1/2}}{x_{1/2}}}\frac{x^{\delta t}_0-x^{\delta t}_{1}}{x_0-x_{1}}}{x_0-x_{1/2}}\\
    &=2\frac{g}{m}\frac{n^{\delta t}\left(\frac{x^{\delta t}_0+x^{\delta t}_1}{2}\right)\textcolor{red}{\frac{x^{\delta t}_0+x^{\delta t}_1}{x_0+x_1}}\frac{x^{\delta t}_0-x^{\delta t}_{1}}{x_0-x_{1}}}{x_0-x_1}\\
    &=\frac{\Phi^+_0}{\textcolor{red}{(x_0+x_1)}(x_0-x_1)(x_0-x_1)}
  \end{split}
\end{equation}

\subsection{$i=N$}

\begin{equation}
  \begin{split}
    \ddot{x}_N &=-\frac{g}{m}\frac{n_{N-1/2}-n_{N+1/2}}{x_{N-1/2}-x_{N+1/2}}\\
    &= -\frac{g}{m}
    \frac{
    n^{\delta t}\left(x^{\delta t}_{N-1/2}\right)\textcolor{red}{\frac{x^{\delta t}_{N-1/2}}{x_{N-1/2}}}\frac{x^{\delta t}_{N-1}-x^{\delta t}_N}{x_{N-1}-x_N}
    -n^{\delta t}\left(x^{\delta t}_{N+1/2}\right)\textcolor{red}{\frac{x^{\delta t}_{N+1/2}}{x_{N+1/2}}}\frac{x^{\delta t}_N}{x_N}}
    {x_{N-1/2}-x_{N+1/2}}\\
    &= -2\frac{g}{m}
    \frac{
    n^{\delta t}\left(\frac{x^{\delta t}_{N-1}+x^{\delta t}_N}{2}\right)\textcolor{red}{\frac{x^{\delta t}_{N-1}+x^{\delta t}_N}{x_{N-1}+x_N}}\frac{x^{\delta t}_{N-1}-x^{\delta t}_N}{x_{N-1}-x_N}
    -n^{\delta t}\left(x^{\delta t}_N/2\right)\textcolor{red}{\frac{x^{\delta t}_N}{x_N}}\frac{x^{\delta t}_N}{x_N}}
    {x_{N-1}}\\
    &= \frac{\Phi^-_N}{\textcolor{red}{(x_{N-1}+x_N)}(x_{N-1}-x_N)x_{N-1}}+\frac{2\frac{g}{m}n^{\delta t}\left(x^{\delta t}_N/2\right)\textcolor{red}{x^{\delta t}_N}x^{\delta t}_N}{\textcolor{red}{x_N}x_Nx_{N-1}}
  \end{split}
\end{equation}


To simplify the code, we will set the numerator of the second fraction as a special value in $\Phi^+_N$

\section{Initial conditions}

The superindex $f^{\delta t}$ indicates $f(t=\delta t)$

\begin{equation}
  x_i^{\delta t}=
  \begin{cases}
    \Delta x\frac{i}{M}        & \text{for}\quad 0\leq i\leq M \\
    \Delta x\left(i-M+1\right) & \text{for}\quad M\leq i\leq N
  \end{cases}
\end{equation}

\begin{equation}
  n^{\delta t}(x)=
  \begin{cases}
    \frac{1}{9}\frac{m}{g}\left[\frac{x-(R_0+2V_\mu\delta t)}{\delta t}\right]^2 & \text{for}\quad R_0-V_\mu\delta t\leq x\leq R_0+2V_\mu\delta t \\
    n(0)                                                                         & \text{for}\quad 0< x\leq R_0-V_\mu\delta t
  \end{cases}
\end{equation}

\begin{equation}
  \dot{x}^{\delta t}(x)=
  \begin{cases}
    2V_\mu+\frac{2}{3}\frac{x-(R_0+2V_\mu\delta t)}{\delta t} & \text{for}\quad R_0-V_\mu\delta t\leq x\leq R_0+2V_\mu\delta t \\
    0                                                         & \text{for}\quad 0< x\leq R_0-V_\mu\delta t
  \end{cases}
\end{equation}


\end{document}
